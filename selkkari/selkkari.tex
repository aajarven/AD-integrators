\documentclass[12pt,a4paper,titlepage]{article}
\usepackage[utf8]{inputenc}
\usepackage[finnish]{babel}
\usepackage{setspace}
\usepackage{parskip}
\usepackage{amssymb}
\usepackage{amsmath}
\usepackage{graphicx}
\usepackage{fancyhdr}
\usepackage[top=1in, bottom=1in, left=1in, right=1in]{geometry}
\usepackage{float}
\usepackage[section]{placeins}
%\usepackage[numbered,autolinebreaks,useliterate]{mcode} % jos tahdot laittaa matlabkoodia näkyville niin kannattaa käyttää tätä

% hyödyllisiä paketteja:
\usepackage{siunitx}\sisetup{per=frac} % SI-yksiköitä.
%\usepackage{supertabular} % jos tarttee isoja taulukoita
%\usepackage{fullpage} % pienemmät marginaalit jos haluaa

\usepackage{hyperref} % lisääthän omat pakettisi ENNEN hyperref'iä
\hypersetup{pdfborder={0 0 0}}
\onehalfspacing
\cfoot{}
\rhead{\thepage}
% asettaa nyk. kappaleen nimen vasempaan ylänurkkaan, saa poistaa jos haluaa
\lhead{\leftmark}

%%%%% kaikki ennen tätä liittyy käytettäviin paketteihin tai dokumentin muotoiluun. siihen ei tarvinne aluksi koskea. %%%%%

%%%%% kansilehti %%%%%
\title{Advanced Dynamics \\ The Plutonian Course Project \vspace{0.5em}}
\author{Anni Järvenpää}
\date{\today}
\begin{document}
\maketitle

% Sisällysluettelo
%\newpage
%\thispagestyle{empty}
%\tableofcontents
%\newpage
%\setcounter{page}{1}
%\parskip=1em \advance\parskip by 0pt plus 2pt
%\pagestyle{fancy}

% prosenttimerkillä alkavat rivit ovat kommentteja: niitä ei katsota dokumenttia käännettäessä eli ne ovat vain kirjoittajaa varten

%%%%%%%%%%%%%%% Oleellinen sisältö alkaa%%%%%%%%%%%%%%%
\section{Teoria}


%\begin{figure}
%\centering
%\includegraphics[width=\textwidth]{yllatyskipsu.jpg} %lveydeksi voi antaa myös vaikkapa 5cm tai muun konkreettisen mitan tai vaihtoehtoisesti asettaa leveydeksi vaikapa 80% tekstin leveydestä parametrilla 0.8\textwidth. Kuvan korjeuden voi asettaa esimerkiksi height=6cm.
%\caption{Kuvatekstissä voisin vaikkapa kertoa, että oikeasti kissa vain haukottelee.}
%\label{kissakuva}
%\end{figure}

\section{Tulokset}
Tarkkoja tuloksia.



%%%%% Sisältö loppuu, lähdeluettelo %%%%%
\bibliographystyle{plain}
\bibliography{selkkarilahteet} %lähdeluettelon tiedot tiedostossa selkkarilahteet.bib. Esimerkiksi helkasta saa kirjojen tiedot valmiiksi bibtex-muodossa, kannattaa hyödyntää.

\appendix
\newpage
\section{Liittyvä liite.} \label{koodi}
Liian laaja leipätekstiin.
\end{document}
